\chapter{Introdução}
\label{cap:introducao}

A automação industrial tem desempenhado um papel fundamental na otimização de processos de produção e na busca por maior eficiência operacional. Nesse contexto, braços robóticos, como os da série Mitsubishi RV-2SDB/RV-2SQB, emergem como ferramentas essenciais para a execução de tarefas complexas com precisão e repetitividade. Contudo, muitas empresas enfrentam desafios para manter esses equipamentos atualizados e funcionais, especialmente quando a manutenção depende de fornecedores externos, o que eleva custos e tempo de inatividade.

Neste trabalho, propõe-se a criação de um software de simulação 3D para braços robóticos, especificamente modelado para o Mitsubishi RV-2SDB/RV-2SQB. A solução, desenvolvida em um ambiente de baixo custo e baseada em bibliotecas open source, visa oferecer uma ferramenta educacional e de treinamento para operadores e estudantes. O destaque principal é a simulação 3D em tempo real, permitindo ao usuário visualizar e acompanhar os movimentos do braço robótico em um ambiente virtual, facilitando o aprendizado e a compreensão da cinemática robótica.

A relevância do projeto reside na possibilidade de oferecer uma ferramenta educacional acessível para o ensino de robótica e cinemática, criando um software de simulação que integra interface gráfica e visualização 3D, ampliando a compreensão das operações robóticas. Além disso, possibilita o compartilhamento e aprimoramento contínuo, já que o código segue uma filosofia open source, beneficiando o meio acadêmico e educacional.

Este documento descreve a fundamentação teórica que embasa o desenvolvimento, a abordagem metodológica adotada, os resultados obtidos e as implicações práticas na educação e na comunidade acadêmica. Espera-se demonstrar que a simulação 3D pode proporcionar maior compreensão, usabilidade e escalabilidade às ferramentas educacionais de robótica.

\section{Motivação}
\label{sec:motivacao}

A motivação central deste trabalho surge da necessidade de uma ferramenta educacional para o ensino de robótica, especificamente para o modelo \textit{Mitsubishi RV-2SDB/RV-2SQB}. A abordagem escolhida foca em soluções de baixo custo e ferramentas open source, criando uma simulação 3D que facilite o aprendizado da cinemática robótica.

Em cenários educacionais, a falta de equipamentos físicos limita o aprendizado prático de robótica. Uma simulação 3D realista que permita visualizar movimentos, testar configurações e compreender a cinemática pode melhorar significativamente o processo de ensino-aprendizagem. A motivação, portanto, combina fatores educacionais, técnicos e pedagógicos, buscando uma solução modular e flexível para o ensino de robótica.

\section{Objetivos}
\label{sec:objetivos}

Os objetivos geral e específicos deste projeto visam oferecer uma solução completa para simulação e ensino do braço robótico Mitsubishi RV-2SDB/RV-2SQB em ambiente educacional. A aplicação integra \textit{software} e interface gráfica, com simulação 3D em tempo real para melhorar visualização e compreensão da cinemática robótica.
\subsection{Objetivo Geral}
\label{sec:objetivo-geral}

Desenvolver um software de simulação 3D robusto \textit{open source} para braços robóticos, permitindo visualização interativa e compreensão da cinemática, além de oferecer uma simulação 3D em tempo real que demonstre os movimentos e comportamentos do robô.

\subsection{Objetivos Específicos}
\label{sec:objetivos-especificos}

Os objetivos específicos incluem modelar e implementar a cinemática de um braço robótico, criar um protocolo universal para comunicação com braços robóticos, 
definindo as transformações matemáticas necessárias para representar e simular fielmente o movimento e funcionamento do braço robôtico Mitsubishi RV-2SDB/RV-2SQB. 
Além disso, pretende-se integrar controles interativos à interface gráfica, de modo a fornecer uma experiência de usuário intuitiva para manipulação da simulação.
O desenvolvimento e incorporação de uma simulação 3D em tempo real permitirá visualização precisa dos movimentos do braço robótico e facilitará a compreensão da cinemática. 
A avaliação do desempenho do sistema em cenários de teste verificará a precisão da simulação, usabilidade da interface e eficiência computacional.
Por fim, o projeto será documentado em formato \textit{open source}, compartilhando bibliotecas e instruções que facilitem a adaptação para outros modelos de braços robóticos ou aplicações educacionais.
