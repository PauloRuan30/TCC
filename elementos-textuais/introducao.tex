\chapter{Introdução}
\label{cap:introducao}

A automação industrial contemporânea, impulsionada pelos paradigmas da Indústria 4.0 \cite{schwab2017}, tem transformado radicalmente as linhas de produção globais. Nesse cenário, a robótica desempenha um papel central na busca por eficiência, precisão e repetibilidade operacional. Manipuladores robóticos, como os da série Mitsubishi RV-2SDB, são amplamente utilizados para tarefas complexas que variam desde a soldagem até a montagem de componentes eletrônicos \cite{craig2017}.



Entretanto, a adoção massiva dessas tecnologias esbarra em um desafio crítico: a formação de mão de obra qualificada. A complexidade matemática e operacional desses equipamentos impõe uma curva de aprendizado íngreme. Além disso, o alto custo de aquisição e manutenção de células robóticas físicas restringe o acesso de estudantes e pesquisadores a esses recursos, criando uma lacuna entre a teoria ensinada em sala de aula e a prática exigida pelo mercado.

Diante desse desafio, a simulação computacional emerge como uma solução viável e segura. Este trabalho propõe o desenvolvimento de um software de simulação 3D voltado especificamente para a modelagem e controle virtual do braço robótico Mitsubishi RV-2SDB. A solução foi construída em um ambiente de baixo custo, utilizando bibliotecas \textit{open source}, com o intuito de democratizar o acesso a ferramentas de treinamento. O diferencial da proposta reside na implementação de uma simulação 3D em tempo real, que permite ao usuário visualizar a cinemática do robô e interagir com seus movimentos em um ambiente virtual seguro, mitigando riscos de danos ao equipamento real durante a fase de aprendizado.

\section{Motivação}
\label{sec:motivacao}

A motivação central deste trabalho origina-se da necessidade de ferramentas de apoio didático que sejam, ao mesmo tempo, tecnicamente precisas e financeiramente acessíveis. Em muitos laboratórios universitários, a relação entre o número de alunos e o número de robôs disponíveis é desproporcional, limitando o tempo de prática individual.

Uma simulação 3D realista atua como um complemento indispensável. Ao mimetizar os comportamentos cinemáticos do robô físico, o software permite que o estudante teste configurações, valide algoritmos de cinemática inversa e compreenda o espaço de trabalho do manipulador antes de enviar comandos para a máquina real \cite{robodk2020}. Portanto, este projeto é motivado pela convergência entre demandas pedagógicas e a aplicação prática de engenharia de software e computação gráfica.

\section{Objetivos}
\label{sec:objetivos}

Esta seção delineia as metas estabelecidas para garantir a eficácia da solução proposta como ferramenta de ensino e simulação.

\subsection{Objetivo Geral}
\label{sec:objetivo-geral}

Desenvolver um software de simulação 3D \textit{open source} para manipuladores robóticos, que permita a visualização interativa em tempo real e facilite a compreensão prática da cinemática direta e inversa do robô Mitsubishi RV-2SDB.

\subsection{Objetivos Específicos}
\label{sec:objetivos-especificos}

Para alcançar o objetivo geral, foram definidos os seguintes objetivos específicos:

\begin{itemize}
    \item Modelar matematicamente a cadeia cinemática do robô utilizando a convenção de Denavit-Hartenberg (D-H), respeitando os limites de juntas e dimensões do modelo real;
    \item Implementar algoritmos de cinemática inversa eficientes, como o FABRIK, capazes de resolver a postura do robô em tempo real;
    \item Desenvolver uma interface gráfica (GUI) intuitiva baseada em ImGui, permitindo o controle granular de cada junta e do efetuador final;
    \item Integrar um sistema de detecção de colisão robusto para garantir o realismo físico da simulação;
    \item Validar o desempenho do sistema comparando os resultados simulados com os dados teóricos do manual do fabricante;
    \item Disponibilizar o código-fonte sob licença livre, fomentando a colaboração acadêmica e a extensão do projeto para outros modelos de robôs.
\end{itemize}