\chapter{Fundamentação Teórica}
\label{cap:fundamentacao-teorica}

A simulação computacional de sistemas robóticos é cada vez mais relevante no contexto educacional e de pesquisa. Avanços em computação gráfica e algoritmos de cinemática permitem criar ambientes virtuais realistas para o estudo de robôs, como o Mitsubishi RV-2SDB/RV-2SQB, sem a necessidade de equipamentos físicos caros, facilitando o aprendizado e reduzindo custos. Este capítulo apresenta os conceitos que sustentam o desenvolvimento do software de simulação proposto, com ênfase nas tecnologias de visualização 3D e modelagem cinemática.

\section{Conceitos de Robótica Industrial}
\label{sec:fundamentacao-teorica-a}

Braços robóticos industriais, como o Mitsubishi RV-2SDB/RV-2SQB, operam com base em cinemática direta e inversa para calcular posições e velocidades das juntas em tarefas precisas. A cinemática direta determina a posição e orientação do efetuador final a partir dos ângulos das juntas, enquanto a cinemática inversa faz o oposto, encontrando os ângulos das juntas para uma dada posição do efetuador. O robô Mitsubishi RV-2SDB/RV-2SQB, com 6 eixos, possui especificações como alcance de ±240° para a junta J1, velocidade máxima de 4.490 mm/s, repetibilidade de ±0.02 mm, e capacidade de carga de 2 kg, conforme o manual técnico. Esses parâmetros são essenciais para a modelagem cinemática no software de simulação.

\section{Tecnologias de Simulação}
\label{sec:fundamentacao-teorica-b}

A integração de bibliotecas gráficas para simulação precisa do robô é fundamental. Bibliotecas open source, como SDL (Simple DirectMedia Layer), facilitam a criação de interfaces interativas, permitindo controle intuitivo da simulação. A interface gráfica, desenvolvida com ImGui, exibe parâmetros do robô e permite ajustes interativos, melhorando a usabilidade educacional.

 \begin{figure}[h!]
    \centering
	\Caption{\label{fig:Figura-1}  Fluxo de dados entre o sistema de simulação 3D e a interface gráfica, ilustrando a interação entre entrada do usuário e visualização virtual.}	
	\UNIFORfig{}{
		\fbox{\includegraphics[width=8cm]{figuras/figura-2}}
	}{
		\Fonte{Elaborado pelo autor}			
	}	
\end{figure}

A Figura~\ref{fig:Figura-1} exibe um diagrama mostrando a interação entre o software de simulação, a interface gráfica, e a visualização 3D, com dados de entrada do usuário e saída visual na simulação.

\section{Simulação 3D em Tempo Real}
\label{sec:fundamentacao-teorica-c}

A simulação 3D em tempo real permite ao usuário visualizar os movimentos do robô em um ambiente virtual, facilitando a compreensão da cinemática e o aprendizado. Utilizando OpenGL para renderização, GLM para cálculos de algebra linear, e Assimp para carregar modelos 3D (e.g.,\textit{rv2sdb.obj}), o software representa fielmente o robô e seus movimentos. A simulação reflete as especificações do robô, como o alcance de ±240° da junta J1, garantindo precisão visual e educacional.

\section{Design de Software}
\label{sec:fundamentacao-teorica-d}

O software foi desenvolvido seguindo princípios de modularidade, com separação de preocupações entre renderização, entrada de dados, cinemática, e interface gráfica. Isso permite manutenção e escalabilidade, facilitando a adaptação para outros robôs ou aplicações educacionais. A arquitetura modular é suportada por bibliotecas como fmt para logging e xmake para gerenciamento de compilação.

 \begin{figure}[h!]
    \centering
	\Caption{\label{fig:Figura-2}  Representação esquemática da arquitetura do software, destacando os módulos de simulação e renderização.}	
	\UNIFORfig{}{
		\fbox{\includegraphics[width=8cm]{figuras/figura-2}}
	}{
		\Fonte{Elaborado pelo autor}			
	}	
\end{figure}