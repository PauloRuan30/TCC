\section{Análise detalhada do sistema de software (síntese técnica)}

A seguir apresenta-se uma análise técnica detalhada do código-fonte do sistema de simulação fornecido.

\subsection{Linguagem e build system}
O código-fonte principal foi implementado em \textbf{C++}, com árvore de diretórios que inclui \texttt{include/} (interfaces e headers) e \texttt{src/} (implementações). O projeto utiliza um sistema de build indicado por \texttt{xmake. lua} e referências a CMake (instruções no README). Para compilação multiplataforma, recomenda-se estabilizar em CMake e oferecer toolchain via conan/docker.

\subsection{Módulos observados}
Foram identificados (entre outros) os seguintes módulos:
\begin{itemize}
  \item \textbf{Renderer: } abstração para renderização 3D (arquivos: \texttt{include/renderer. h}, \texttt{src/renderer. cpp}), responsável por desenhar malhas, câmera e iluminação.
  \item \textbf{Model Loader: } carregamento de modelos 3D (GLB/GLTF) e assets (arquivos em \texttt{assets/}).
  \item \textbf{Animation Controller: } controlador de animações e interpolação de keyframes.
  \item \textbf{Input Handler / GUI Manager: } gerenciamento de eventos de usuário e interface.
  \item \textbf{Performance Monitor: } coleta de métricas de desempenho (FPS, tempos de frame).
\end{itemize}

\subsection{Fluxo de execução (runtime)}
A execução segue o padrão clássico de aplicações gráficas:
\begin{enumerate}
  \item inicialização do motor gráfico e carregamento de recursos (malhas, texturas, cenas);
  \item loop principal: leitura de eventos de input, atualização de estado (simulação e animação) e chamada ao renderer para desenhar o frame;
  \item coleta de métricas e, eventualmente, gravação de logs.
\end{enumerate}

\subsection{Pontos fortes e pontos fracos}
\textbf{Pontos fortes: }
\begin{itemize}
  \item arquitetura modular com separação de responsabilidades (render, input, assets);
  \item inclusão de assets prontos para demonstração (ex. : arquivos GLB em \texttt{assets/}).
\end{itemize}

\textbf{Pontos a melhorar: }
\begin{itemize}
  \item falta de scripts de build padrão documentados para todas as plataformas;
  \item ausência de testes automatizados e benchmarks padronizados;
  \item documentação técnica parcial: recomenda-se complementar com diagramas (UML/arquitetura) e exemplos de uso passo a passo.
\end{itemize}
