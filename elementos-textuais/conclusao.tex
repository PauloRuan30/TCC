\chapter{Conclusões e Trabalhos Futuros}
\label{chap:conclusoes}

\section{Conclusões Gerais}

O desenvolvimento do software de simulação 3D para braços robóticos atingiu com êxito os objetivos propostos, resultando em uma ferramenta funcional capaz de auxiliar no processo de ensino-aprendizagem de robótica industrial. Foi possível criar uma aplicação \textit{open source}, de baixo custo e com visualização em tempo real, que simula com precisão a cinemática direta e inversa do modelo Mitsubishi RV-2SDB.

Os testes realizados demonstraram que a integração entre os algoritmos matemáticos e a renderização gráfica foi bem-sucedida. O sistema manteve uma taxa de quadros (FPS) superior a 120 em resolução Full HD, garantindo uma experiência de uso fluida. Além disso, a implementação do algoritmo FABRIK provou-se eficaz para a resolução da cinemática inversa, apresentando erro de posicionamento inferior a 1 mm em 98\% dos casos testados, o que é aceitável para fins educacionais. A validação com usuários indicou que a interface baseada em ImGui é intuitiva, permitindo que estudantes compreendam os efeitos da manipulação das juntas no espaço cartesiano de forma imediata.

\section{Contribuições do Trabalho}
\label{sec:contribuicoes}

Além do produto de software em si, este trabalho oferece contribuições acadêmicas e práticas relevantes:

\begin{itemize}
  \item \textbf{Democratização do Ensino:} Disponibilização de uma ferramenta gratuita que mitiga a necessidade de hardware físico caro para as etapas iniciais do aprendizado de robótica.
  \item \textbf{Didática Algorítmica:} A implementação modular serve como material de estudo para disciplinas de Computação Gráfica e Robótica, demonstrando na prática a aplicação de conceitos de Álgebra Linear.
  \item \textbf{Integração Tecnológica:} Demonstração da viabilidade de construir simuladores complexos utilizando apenas bibliotecas livres (OpenGL, SDL, ImGui), sem dependência de \textit{engines} comerciais pesadas.
  \item \textbf{Base para Pesquisa:} O código estruturado permite que outros pesquisadores expandam o sistema para testar novos algoritmos de controle ou planejamento de trajetória.
\end{itemize}

\section{Limitações do Sistema}
\label{sec:limitacoes}

Apesar dos resultados satisfatórios, o sistema apresenta limitações inerentes ao escopo do projeto. A principal delas é a ausência de um modelo dinâmico completo; ou seja, a simulação considera apenas a geometria do movimento (cinemática), ignorando forças como inércia, atrito e gravidade, o que impede o uso para testes de carga ou controle de torque.

Adicionalmente, o software opera atualmente como uma simulação puramente virtual ("offline"), sem comunicação direta com o controlador do robô físico via TCP/IP. A compatibilidade com outros modelos de robôs também é restrita, uma vez que não há ainda um importador universal para formatos de descrição de robôs como URDF (\textit{Unified Robot Description Format}).

\section{Trabalhos Futuros}
\label{sec:trabalhos-futuros}

Para a continuidade e evolução deste projeto, sugerem-se as seguintes linhas de desenvolvimento:

\begin{itemize}
  \item \textbf{Integração Hardware-in-the-Loop (HIL):} Implementar protocolos de comunicação industrial para que o simulador possa controlar o robô físico real ou receber dados de telemetria dele em tempo real.
  \item \textbf{Planejamento de Trajetória:} Incorporar algoritmos de planejamento de movimento, como RRT (\textit{Rapidly-Exploring Random Tree}) e PRM (\textit{Probabilistic Roadmap}) \cite{rrt2000}, permitindo que o robô desvie autonomamente de obstáculos.
  \item \textbf{Suporte a URDF:} Desenvolver um parser para arquivos URDF, tornando o simulador agnóstico ao modelo do robô e ampliando seu uso para qualquer manipulador industrial ou colaborativo.
  \item \textbf{Realidade Estendida (XR):} Portar a visualização para frameworks como OpenXR, possibilitando o uso de óculos de Realidade Virtual para uma imersão completa no ambiente de treinamento.
  \item \textbf{Sim2Real e Machine Learning:} Utilizar o ambiente simulado para gerar dados sintéticos visando o treinamento de redes neurais para tarefas de \textit{grasping} (preensão) ou visão computacional.
\end{itemize}

\section{Considerações Finais}

Este trabalho reforça a tese de que é possível alinhar rigor técnico e acessibilidade no desenvolvimento de software educacional. Ao integrar computação gráfica de alto desempenho com fundamentos sólidos de robótica, a ferramenta desenvolvida oferece uma base promissora para a formação da próxima geração de engenheiros e cientistas da computação. Com as expansões propostas, o sistema tem potencial para alcançar maturidade comparável a ferramentas industriais, servindo não apenas ao ensino, mas também à pesquisa avançada em robótica.