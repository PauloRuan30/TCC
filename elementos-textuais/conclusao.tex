\chapter{Conclusões e Trabalhos Futuros}
\label{chap:conclusoes-e-trabalhos-futuros}

\section{Conclusões Gerais}


O desenvolvimento do software de simulação 3D para braços robóticos atingiu com êxito os objetivos propostos neste trabalho. Foi possível criar uma ferramenta educacional open source, de baixo custo e com visualização em tempo real, capaz de simular com precisão a cinemática direta e inversa do robô Mitsubishi RV-2SDB.


Os principais resultados obtidos incluem:


\begin{itemize}


  \item Implementação funcional da cinemática direta e inversa com erro inferior a 1 mm;


  \item Simulação estável e fluida com FPS superior a 120 em ambiente Full HD;


  \item Detecção de colisões eficiente por meio dos algoritmos GJK \cite{gjk1988} e EPA;


  \item Interface interativa, intuitiva e leve, baseada em ImGui;


  \item Avaliação positiva de estudantes e docentes quanto à utilidade didática.


\end{itemize}


\section{Contribuições do Trabalho}
\label{sec:contribuicoes-do-trabalho}


As principais contribuições do presente trabalho são:


\begin{itemize}

  \item Prototipagem de um simulador open source voltado ao ensino de robótica industrial;

  \item Integração entre computação gráfica, álgebra linear e robótica em uma aplicação prática;

  \item Implementação didática de algoritmos de cinemática e colisão, com código modular e extensível;

  \item Criação de um ambiente educacional acessível, que permite explorar conceitos complexos de forma visual e interativa;

  \item Base para pesquisas futuras em simulação robótica, podendo ser expandida para controle físico ou aprendizado de máquina.

\end{itemize}

\section{Limitações do Sistema}

Apesar dos resultados positivos, algumas limitações foram identificadas:

\begin{itemize}

  \item Ausência de integração direta com o robô físico Mitsubishi RV-2SDB (simulação puramente virtual);

  \item Não implementação de controle dinâmico (apenas cinemática);

  \item Falta de suporte a modelos CAD externos (URDF, STL múltiplos);

  \item Interface sem recursos avançados de realidade aumentada (AR/VR).

\end{itemize}

\section{Limitações}
\label{sec:limitacoes}


\section{Trabalhos Futuros}
\label{sec:trabalhos-futuros}


Para continuidade e evolução do projeto, sugerem-se as seguintes direções:

\begin{itemize}

  \item Integração Hardware-in-the-Loop (HIL): permitir que o simulador controle o robô físico por meio de protocolo TCP/IP, validando movimentos reais.

  \item Planejamento de Trajetória: adicionar algoritmos como RRT \cite{rrt2000} (Rapidly-Exploring Random Tree) e PRM (Probabilistic Roadmap).

  \item Suporte a múltiplos robôs: implementação de gerenciamento de instâncias e sincronização de movimentos em ambiente colaborativo.

  \item Exportação e Importação de Modelos CAD (URDF/STL): ampliar compatibilidade com outros robôs industriais.

  \item Integração com Realidade Aumentada e Virtual (AR/VR): utilização de frameworks como OpenXR para exibição tridimensional imersiva.

  \item Implementação Web-based (WebGL): disponibilizar o simulador em navegadores, facilitando o acesso remoto.

  \item Módulo de aprendizado de máquina (Sim2Real): treinar modelos de rede neural para prever configurações de juntas e evitar colisões.

\end{itemize}


\section{Considerações Finais}

O presente trabalho demonstrou que é possível desenvolver um simulador 3D educacional robusto, eficiente e de código aberto, integrando computação gráfica e robótica de forma acessível. O projeto atingiu os objetivos propostos e oferece base sólida para aplicações futuras tanto em ensino quanto em pesquisa.

Com as melhorias propostas, o software poderá alcançar maturidade comparável a ferramentas industriais consolidadas, contribuindo significativamente para a formação de estudantes e pesquisadores na área de Robótica e Computação Aplicada.




