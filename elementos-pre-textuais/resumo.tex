Este trabalho apresenta o desenvolvimento de uma Interface Homem-Máquina (IHM) para o controle de um braço robótico, propondo uma solução de hardware e software com joystick e interface gráfica para simulação de movimentos em tempo real em um ambiente computacional. O objetivo central é viabilizar o reparo e aprimoramento de equipamentos industriais legados, por meio de uma abordagem de baixo custo e código aberto. Além de fornecer controle manual (via joystick), a interface implementa a possibilidade de ajustes automáticos, permitindo o retrofitting de maneira acessível e flexível. Ao longo do texto, aborda-se a justificativa do projeto, a metodologia utilizada, os resultados práticos e as perspectivas de aplicação em larga escala.

\textbf{Palavras-chave}: Automação Industrial. Controle de Robôs. Interface Gráfica. Simulação 3D. Retrofit.